\chapter*{Список сокращений и условных обозначений} % Заголовок
\addcontentsline{toc}{chapter}{Список сокращений и условных обозначений}  % Добавляем его в оглавление
\noindent
%\begin{longtabu} to \dimexpr \textwidth-5\tabcolsep {r X}
\begin{longtabu} to \textwidth {r X}
% Жирное начертание для математических символов может иметь
% дополнительный смысл, поэтому они приводятся как в тексте
% диссертации

\textbf{БОЭП} & бортовая оптико-электронная система \label{acroAEOS}\\

\textbf{СС} & Следящая система  \\

\textbf{ОЭП (ОЭС)} & Оптико-электронные приборы и системы \label{acroEOS}\\

\textbf{КИМ} & компьютерная имитационная модель \label{acroCSM} \\

\textbf{АЧХ (АФЧХ)} & Амплитудно-частотно фазовая характеристика \\

\textbf{СА} & система амортизации \\

\textbf{САФ} & система автоматической фокусировки \\

\textbf{МК} & микроконтроллер \\

\textbf{ПЛИС} & программируемая логическая интегральная схема \\

\textbf{ФПУ (ФПМ)} & фотоприёмное устройство (матрица) \\

\textbf{ДУС} & датчик угловых скоростей \\
\textbf{ДУП} & датчик углового положения (Энкодер) \label{acroDUP} \\

\textbf{ДТВ} & датчик телевизионный \\

\textbf{ГСН} & головкой самонаведения \label{acroGSN} \\

\textbf{ИБ} & измерительный блок \\

\textbf{БВМ} & бортовой вычислительная машина \\

\textbf{УМ} & усилитель мощности \\

\textbf{ВЧК} & модель высокочастотных колебаний \\

\textbf{ИИ} & источник излучения \\

\textbf{ИК} & инфракрасный \\

\textbf{ИМ} & имитационная модель \\

\textbf{КИ} & качество изображения \\

\textbf{КИМ} & компьютерная имитационная модель \\

\textbf{КМ} & компьютерная модель \\

\textbf{ЛА} & летательный аппарат \label{acroLA} \\

\textbf{ММ} & математическая модель \\

\textbf{МФП} & матричный фотоприемник \\

\textbf{НЧД} & модель низкочастотного движения \\

\textbf{НУТВ} & телевизионный \\

\textbf{ОН} & объект наблюдения \label{acroON}\\

\textbf{ОПФ} & оптическая передаточная функция \\

\textbf{ОС} & оптическая система \\

\textbf{ОУ} & объект управления \\

\textbf{ОЭК} & оптико-электронный комплекс \\

\textbf{ОЭС} & оптико-электронная система \\

\textbf{ПЗС} &прибор с зарядовой связью \\

\textbf{ПЗРК} & пусковой зенитный ракетный комплекс \\

\textbf{ПИД} & пропорционально-интегрально-дифференциальный \\

\textbf{ПСт} & система прецизионной стабилизации \\

\textbf{САУ} & система автоматического управления \label{acroSAU}\\

\textbf{САФ} & система автоматической фокусировки \\

\textbf{СВ} & система виброзащиты \\

\textbf{СКИ} & системы обеспечения качества изображения \\

\textbf{СОЭП} & система оптико-электронного подавления  \label{acroSOEP} \\

\textbf{ССк} & система сканирования \\

\textbf{ССл} & система слежения \\

\textbf{СТР} & система терморегулрования \\

\textbf{ТЗ} & техническое задание \label{acroTZ} \\

\textbf{ТС}  & телевизионная система \label{acroTS} \\

\textbf{УВ} & устройство визуализации \\

\textbf{УР} & управляемая ракета \\

\textbf{УПУ} & усилительно-преобразовательное устройство \\

\textbf{ЦСУ} & цифровая система управления \\

\textbf{ФКЯ} & фотоприемник на квантовых ямах \\

\textbf{ФПМ} & функция передачи модуляции \label{acroFPM}\\

\textbf{ФЦО} & фоноцелевая обстановка \\

\textbf{ТВС} & тепловизионной системы \label{acroTVS} \\

\textbf{ФС} & фотографической системы \label{acroFS} \\

\textbf{НИР} & научно исследовательской работы \label{acroNIR} \\

\textbf{БЛА} & беспилотные летательные аппараты \label{acroUAV} \\

\textbf{ПВО} & средства противовоздушной обороны \label{acroPVO} \\

\textbf{ТТХ} &  тактико технические требования \label{acroTTX} \\

\textbf{ТП} & технического предложения \\

\textbf{КД} & конструкторской документации \\

\textbf{НАТО} & Североатлантический Альянс \label{acroNATO}\ \\

\textbf{СВ} & вооруженных сил \\

\textbf{ВМС} & военно-морских сил \\

\textbf{ЦУ} & целеуказания  \\

\textbf{ТпВ} & тепловизор \\

\textbf{GPS} & система глобального позиционирования \\

\textbf{ЭВМ} &  электронно вычислительная машина \\




\end{longtabu}
\addtocounter{table}{-1}% Нужно откатить на единицу счетчик номеров таблиц, так как предыдующая таблица сделана для удобства представления информации по ГОСТ
