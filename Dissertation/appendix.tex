\chapter{Примеры вставки листингов программного кода} \label{app:A}

Для крупных листингов есть два способа. Первый красивый, но в нём могут быть
проблемы с поддержкой кириллицы (у вас может встречаться в~комментариях
и печатаемых сообщениях), он представлен на листинге~\ref{lst:hwbeauty}.
\begin{ListingEnv}[!h]% настройки floating аналогичны окружению figure
    \captiondelim{ } % разделитель идентификатора с номером от наименования
    \caption{Программа ,,Hello, world`` на \protect\cpp}
    % далее метка для ссылки:
    \label{lst:hwbeauty}
    % окружение учитывает пробелы и табуляции и применяет их в сответсвии с настройками
    \begin{lstlisting}[language={[ISO]C++}]
	#include <iostream>
	using namespace std;

	int main() //кириллица в комментариях при xelatex и lualatex имеет проблемы с пробелами
	{
		cout << "Hello, world" << endl; //latin letters in commentaries
		system("pause");
		return 0;
	}
    \end{lstlisting}
\end{ListingEnv}%
Второй не~такой красивый, но без ограничений (см.~листинг~\ref{lst:hwplain}).
\begin{ListingEnv}[!h]
    \captiondelim{ } % разделитель идентификатора с номером от наименования
    \caption{Программа ,,Hello, world`` без подсветки}
    \label{lst:hwplain}
    \begin{Verb}

        #include <iostream>
        using namespace std;

        int main() //кириллица в комментариях
        {
            cout << "Привет, мир" << endl;
        }
    \end{Verb}
\end{ListingEnv}



\section{Стандартные префиксы ссылок} \label{app:B4}

Общепринятым является следующий формат ссылок: \texttt{<prefix>:<label>}.
Например, \verb+\label{fig:knuth}+; \verb+\ref{tab:test1}+; \verb+label={lst:external1}+.
В таблице \ref{tab:tab_pref} приведены стандартные префиксы для различных типов ссылок.

\begingroup
    \centering
    % \small
    \begin{longtable}[c]{|c|c|}
    \caption{Стандартные префиксы ссылок}%
    \label{tab:tab_pref}% label всегда желательно идти после caption
    \\[-0.45\onelineskip]
    \hline
    \textbf{Префикс} & \textbf{Описание} \\ \hline
    \endfirsthead%
    \caption*{\tabcapalign Продолжение таблицы~\thetable}\\[-0.45\onelineskip]
    \hline
    \textbf{Префикс} & \textbf{Описание} \\ \hline
    \endhead
    \hline
    \endfoot
    \hline
    \endlastfoot
    ch:     & Глава             \\
    sec:    & Секция            \\
    subsec: & Подсекция         \\
    fig:    & Рисунок           \\
    tab:    & Таблица           \\
    eq:     & Уравнение         \\
    lst:    & Листинг программы \\
    itm:    & Элемент списка    \\
    alg:    & Алгоритм          \\
    app:    & Секция приложения \\
    \end{longtable}
% \normalsize% возвращаем шрифт к нормальному
\endgroup

Для упорядочивания ссылок можно использовать разделительные символы.
Например, \verb+\label{fig:scheemes/my_scheeme}+ или \\ \verb+\label{lst:dts/linked_list}+.

\section{Очередной подраздел приложения} \label{app:B5}

Нужно больше подразделов приложения!

\section{И ещё один подраздел приложения} \label{app:B6}

Нужно больше подразделов приложения!
