\chapter*{Словарь терминов}             % Заголовок
\addcontentsline{toc}{chapter}{Словарь терминов}  % Добавляем его в оглавление

\noindent
%\begin{longtabu} to \dimexpr \textwidth-5\tabcolsep {r X}
\begin{longtabu} to \textwidth {r X}
	% Жирное начертание для математических символов может иметь
	% дополнительный смысл, поэтому они приводятся как в тексте
	% диссертации
	$E_2 = \left( \begin{array}{c c}
	1 & 0 \\
	0 & 1
	\end{array}\right)$ 
	& единичная матрица второго порядка \\
	$q=\left( \begin{array}{c}
	\alpha \\
	\beta
	\end{array}\right)$ & обобщенная координата ОУ\\
	$\alpha$ & угол поворота ротора азимутального привода\\
	$\beta$ & угол поворота ротора угломестного привода\\
	$\omega = \left( \begin{array}{c}
\dot \psi \\
\dot \vartheta
\end{array}\right)$ & обобщенная угловая скорость ЛА\\
	$\psi$ & положение вертолета в азимутальной плоскости\\
	$\vartheta$ & положение вертолета в угломестной плоскости\\
	$K_{c1} (K_{c\alpha})$ & добротность канала азимута\\
	$K_{c2} (K_{c\beta})$ & добротность канала угла места\\
	
	$\varDelta L$ & запас устойчивости по амплитуде\\
	$\varDelta \varphi$ & запас устойчивости по фазе\\
	
	\(  \omega _{\textit{ср}}\) & частота среза непрерывной системы\\ \(  \omega _{\textit{гр}}\) & граничная частота\\
	
	$\begin{rcases}
	a_n\\
	b_n
	\end{rcases}$  & 
	коэффициенты разложения Ми в дальнем поле соответствующие
	электрическим и магнитным мультиполям ещё раз, но~без окружения
	minipage нет вертикального выравнивания по~центру.
	\\
	$j$ & тип функции Бесселя\\
	$k$ & волновой вектор падающей волны\\
	
	$\begin{rcases}
	a_n\\
	b_n
	\end{rcases}$  & 
	\begin{minipage}{\linewidth}
		\vspace{0.7em}
		и снова коэффициенты разложения Ми в дальнем поле соответствующие
		электрическим и магнитным мультиполям, теперь окружение minipage есть
		и добавлено много текста, так что описание группы условных
		обозначений значительно превысило высоту этой группы... Для отбивки
		пришлось добавить дополнительные отступы.
		\vspace{0.5em}
	\end{minipage}
	\\
	$L$ & общее число слоёв\\
	$l$ & номер слоя внутри стратифицированной сферы\\
	$\lambda$ & длина волны электромагнитного излучения
	в вакууме\\
	$n$ & порядок мультиполя\\
	$\begin{rcases}
	{\mathbf{N}}_{e1n}^{(j)}&{\mathbf{N}}_{o1n}^{(j)}\\
	{\mathbf{M}_{o1n}^{(j)}}&{\mathbf{M}_{e1n}^{(j)}}
	\end{rcases}$  & сферические векторные гармоники\\
	$\mu$  & магнитная проницаемость в вакууме\\
	$r,\theta,\phi$ & полярные координаты\\
	$\omega$ & частота падающей волны\\
	
	\textbf{BEM} & boundary element method, метод граничных элементов\\
	
\end{longtabu}
\addtocounter{table}{-1}% Нужно откатить на единицу счетчик номеров таблиц, так как предыдующая таблица сделана для удобства представления информации по ГОСТ

\begin{equation}
\begin{aligned}

\end{aligned}
\end{equation} - единичная матрица


