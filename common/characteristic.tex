
{\actuality} \textbf{Актуальность темы.} Авиационные и космические оптико-электронные системы ( \hyperref[acroEOS]{ОЭС}) нашли широкое применение при выполнении задач наблюдения, подсветки, целеуказания, в том числе при решении народнохозяйственных задач и задач обороны и безопасности. Такие приборы используются в разработках предприятиями ОАО НПО ГИПО, ОАО ПО УОМЗ, ФГУП НИИ Полюс, АО Швабе, ОАО НПО Карат, АО Стелла-К и др. 

Среди множества приборов, решающих указанные задачи, рассматривается класс бортовых оптико-электронных систем, работающих в режимах кругового обзора пространства и слежения. Известно, что движение носителя существенно ухудшает качество изображения для смотрящих систем и точность позиционирования для излучающих. Это ограничивает применение разработанных оптико-электронных систем и их моделей, в частности при проектировании бортовых оптико-электронных систем кругового обзора и слежения, в которых динамика подсистем оказывает значительное влияние на качество изображения. По этим причинам и с учетом необходимости исследования динамики бортовых оптико-электронны систем возникает актуальная задача построения адекватных математических (аналитических) и имитационных компьютерных моделей, учитывающих динамику движения управляемых \hyperref[acroEOS]{ОЭС} рассматриваемого класса, и исследования влияния параметров системы на ее динамические свойства и совершенствования характеристик \hyperref[acroEOS]{ОЭС}.

При проектировании современных бортовых  \hyperref[acroEOS]{ОЭС} широко используется метод компьютерного моделирования  \hyperref[acroEOS]{ОЭС} (Ю.Г. Якушенков, В.В. Тарасов, И.П. Торшина, В.П. Иванов, В.А. Овсянников). Компьютерное моделирование позволяет решать задачи рационального выбора структуры, параметров, элементной базы  \hyperref[acroEOS]{ОЭС}, обеспечивающих требуемые показатели эффективности при заданных ограничениях и позволяет не проводить в ряде случаев дорогостоящие натурные исследования и испытания на этапе проектирования  \hyperref[acroEOS]{ОЭС}. Методы компьютерного моделирования  \hyperref[acroEOS]{ОЭС} основываются на работах Дж. Ллойда, М.М. Мирошникова, А.В. Демина, В.И. Воронова, В.А. Балоева, В.С. Яцыка, Р.М.  Алеева и их учеников. Как правило, компьютерные модели  \hyperref[acroEOS]{ОЭС} предназначены для  \hyperref[acroEOS]{ОЭС} конкретного назначения. 

Известно, что динамика движения объекта наблюдения и носителя сильно влияет на качество изображения и вероятность удержания объекта наблюдения на оптической оси, следящей  \hyperref[acroEOS]{ОЭС}, а для излучающих оптико-электронных систем ухудшает точность позиционирования и вероятность выполнения боевой задачи.  Как правило для обеспечения удержания цели в поле зрения или в секторе облучения необходимо захватить довольно большой телесный угол обзора что приводит к увеличению габаритов оптической системы, а в случае с излучающими системами и увеличение габаритов и мощностей силовой электроники. По этим причинам и возникает актуальная задача увеличения точности наведения и удержания цели, что неизбежно ведет за собой необходимость уточнения математической модели, дополнение её исследованиями, затрагивающими динамическую составляющую, как носителя, так и самого прибора. Кроме того, более точная математическая модель позволяет в дальнейшем решать задачу оптимизации характеристик прибора по требуемым критериям.

\ifsynopsis
Этот абзац появляется только в~автореферате.
\else
Этот абзац появляется только в~диссертации.
\fi

% {\progress} j
% Этот раздел должен быть отдельным структурным элементом по
% ГОСТ, но он, как правило, включается в описание актуальности
% темы. Нужен он отдельным структурынм элемементом или нет ---
% смотрите другие диссертации вашего совета, скорее всего не нужен.

{\aim} работы является улучшение массово-габаритных характеристик бортовых  \hyperref[acroEOS]{ОЭС}, повышение качества изображения для следящих систем и точности позиционирования для излучающих систем, улучшение динамических характеристик бортовых автоматических оптико-электронных систем в режимах наведения и стабилизации за счет рационального выбора их параметров при компьютерном моделировании.

Для~достижения поставленной цели необходимо было решить следующие {\tasks}:
\begin{enumerate}
  \item \todo{Исследовать, разработать, вычислить и~т.\:д. и~т.\:п.}
  \item Исследовать, разработать, вычислить и~т.\:д. и~т.\:п.
  \item Исследовать, разработать, вычислить и~т.\:д. и~т.\:п.
  \item Исследовать, разработать, вычислить и~т.\:д. и~т.\:п.
\end{enumerate}


{\novelty} заключается в разработке методики построения математических и компьютерных моделей, учитывающих движение летательного аппарата ( \hyperref[acroLA]{ЛА}) и динамику подсистем бортовых  \hyperref[acroEOS]{ОЭС} наведения и слежения на основе оценки качества их изображения и точности позиционирования для излучающих систем. Решение научной задачи проводилось по следующим направлениям:
\begin{enumerate}
  \item Разработка методики построения математических и компьютерных моделей с применением информационных технологий, учитывающих движение  \hyperref[acroLA]{ЛА}, в виде совокупности взаимосвязанных моделей подсистем автоматических  \hyperref[acroEOS]{ОЭС} и атмосферы, а также динамических моделей их подсистем, обеспечивающих требуемое качество изображения, и исследования их динамических свойств (Глава \ref{ch:ch2}). 
  \item Разработка адекватных математических (Глава \ref{ch:ch3}) и имитационных компьютерных моделей (Глава \ref{ch:ch5}) управляемого бортового  \hyperref[acroEOS]{ОЭП}, учитывающих требования  \hyperref[acroTZ]{ТЗ}, действующие возмущения, движение носителя, динамику систем обеспечения (регулирования) качества изображения, а также особенности конструкции системы. 
  \item Синтез регулятора в режимах наведения и стабилизации с применением информационных технологий (Глава \ref{ch:ch4}). Разработка электроники и программного обеспечения реализующее синтезированный регулятор.
\end{enumerate}

{\influence} 

Разработанная методика и компьютерная имитационная модель позволяют проводить комплексные исследования на этапах предварительных разработок (эскизное проектирование), модернизации и испытаний, бортовых  \hyperref[acroEOS]{ОЭС}, а также проводить рациональный выбор параметров подсистем, сократить трудоемкость и сроки этапов разработки, исследования, настройки и отладки  \hyperref[acroEOS]{ОЭС}.

Выработаны рекомендации по выбору основных параметров подсистем для конкретного варианта расположения и конструкции бортовой  \hyperref[acroEOS]{ОЭС}.

{\methods} \todo{методы исследования}\ldots

{\defpositions}
\begin{enumerate}
  \item \textit{Методика разработки математических и компьютерных моделей бортовых  \hyperref[acroEOS]{ОЭС}}, позволяющая проводить имитационное моделирование, выбор параметров и исследование бортовых  \hyperref[acroEOS]{ОЭС} в процессе разработки, создания и испытания в соответствии с требованиями  \hyperref[acroTZ]{ТЗ} и учетом конструктивных ограничений. 
  \item \textit{Математическая модель  \hyperref[acroAEOS]{Б \hyperref[acroEOS]{ОЭП}}} включающая в себя кинематическую и механическую схемы, представляющую собой уравнения динамической модели  \hyperref[acroSAU]{САУ} как два абсолютно твердых тела закрепленных на  \hyperref[acroLA]{ЛА}, электрическую модель привода и их линеаризацию.
  \item \textit{Разработанные алгоритмы управления}, синтезированные на основании технического задания в рамках научно исследовательской работы (НИР) частотным методом \cite[]{Babaev}.
  \item \textit{Разработанная КИМ пространственной модели и исследование динамики управления  \hyperref[acroEOS]{ОЭП}} по азимуту и углу места в режимах наведения и стабилизации при действии возмущений, идущих от носителя и движения объекта наблюдения.
\end{enumerate}

{\reliability} \todo{полученных результатов обеспечивается Результаты находятся в соответствии с результатами, полученными другими авторами.}


{\probation}
Результаты диссертационной работы в виде компьютерной имитационной модели и макета бортовой  \hyperref[acroEOS]{ОЭС} использованы на предприятии АО «Стелла-К» при модернизации вертолетных систем постановки оптических помех, что подтверждается соответствующим актом.

Методика построения математических и имитационных моделей и исследования автоматических  \hyperref[acroEOS]{ОЭС} используется в учебном процессе.

{\contribution} \todo{Автор принимал активное участие} \ldots

{\publications} Основные результаты по теме диссертации изложены в XX печатных изданиях,
    X из которых изданы в журналах, рекомендованных ВАК,
    X "--- в тезисах докладов.
    
\begin{enumerate}
  \item Бурдинов К.А., Карпов А.И., Кренев В.А.   Методика разработки и исследование динамики систем автоматического управления бортовыми комплексированными оптико-электронными приборами с применением компьютерных технологий // Труды XI Международной Четаевской конференции «Аналитическая механика, устойчивость и управление», Том 3, секция 3 «Управление», часть I, с. 157-161.
  \item Герасин Е.А., Карпов А.И., Кренев В.А., Бурдинов К.А., Соколов О.В., Лукин К.Г. Разработка и исследование алгоритмов управления оптико-электронным прибором вертолетного базирования // Сборник докладов Всероссийской научно-практической конференции с международным участием «Новые технологии, материалы и оборудование Российской авиакосмической отрасли» (АКТО-2016), с.430-437.
  \item К.А. Бурдинов, А.И. Карпов, В.А. Кренев. Методика разработки и исследование динамики систем виброзащиты и управления бортовыми оптико-электронными приборами с применением компьютерных технологий // Вестник Казанского государственного технического университета им. А.Н. Туполева. - 2018. - №2, 6 с.
  \item К.А. Бурдинов*, В.А. Туранов. Проектирование системы высотной видеосъемки. Прикладная электродинамика, фотоника и живые системы. Материалы Международной научно-технической конференции молодых ученых, аспирантов и студентов, ISBN 978-5-9907911-0-7
Казань. Изд-во ООО «16ПРИНТ» 2016. 175-179с.
  \item К.А. Бурдинов, Р.А. Мирзин, К.Э Миллер. Проектирование конструкции гиро платформы для видеокамеры. Материалы Международной научно-технической конференции молоды ученых, аспирантов и студентов. Прикладная электродинамика, фотоника и живые системы., ISBN 978-5-9907911-0-7. Казань. Изд-во ООО «16ПРИНТ» 2016. 199-200с.
  \item К.А.Бурдинов, А.Е. Смирнов. К задаче моделирования динамики формирования поверхности составного параболического зеркала. Материалы Международной молодежной научной конференции «XXXIII Гагаринские чтения» апрель М. МАТИ. 2012. 136-141с.
  \item Д.А. Молин, А.Е. Смирнов, К.А. Бурдинов. К задаче моделирования процесса формирования поверхности составного параболического зеркала и оценке качества изображения. // Труды X Международной Четаевской конференции «Аналитическая механика, устойчивость и управление», Том 1, секция 3 «Аналитическая механика, часть I, 2012. 137-145 с.
  \item К.А. Бурдинов. Исследование динамики управления аэрофотоаппаратом и оценка влияния динамических погрешностей систем. обеспечивающих его работу, на качество изображения. Материалы Международной молодежной научной конференции «XIX Туполевские чтения» апрель Казань. КНИТУ- КАИ. 2011. 30-31с.
  \item Бурдинов К.А., Смирнов А. Е. Реализация алгоритмов управления моментным двигателем  \hyperref[acroEOS]{ОЭП} на ПЛИС. ХХII Туполевские чтения: Межд. молодежная научная конференция, 19-21 октября 2015г.: материалы конференции. Сборник докладов. Казань: Изд-во «Фолиант» 2015, том 2, 60-63с.
  \item Бурдинов К.А., Смирнов А. Е. Разработка и исследование макета системы амортизации бортового оптико-электронного прибора с лазерным каналом Голография. Наука и практика: сборник трудов 12-й Межд. конференции «ГолоЭкспо-2015». Казань: Изд-во КНИТУ-КАИ, 2015. - 387-389с.
  \item Бурдинов К.А.; Смирнов А.Е.; Карпов А.И. Разработка конструкции управляемого подвеса лазерного канала следящего оптико-электронного прибора в solid works. Материалы Международной научно-технической конференции молоды ученых, аспирантов и студентов. Прикладная электродинамика, фотоника и живые системы., ISBN 978-5-9907911-0-7. Казань. Изд-во ООО «16ПРИНТ» 2016. 279-284с.
  \item Карпов А.И., Бурдинов К.А. Динамика и качество изображения видеокамеры, установленной на квадракоптере. Голография. Наука и практика: сборник трудов 12-й Межд. конференции «ГолоЭкспо-2015». Казань: Изд-во КНИТУ-КАИ, 2015. - 387-389с.
  \item К.А. Бурдинов, А.И. Карпов, В.А. Кренев. Методика разработки и исследования динамики систем виброзащиты и управления бортовыми оптико-электронными приборами с применением компьютерных технологий. //Вестник КНИТУ-КАИ. Том 2, с.152-161. Казань 2018
  \item 
\end{enumerate}
